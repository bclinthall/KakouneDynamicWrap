\documentclass[12pt]{proposal} 

\usepackage{natbib}


%metadata
\title{Automatic prediction of prosodic stress in psalm text}
\degree{MASTER OF SCIENCE}
\degreeabbrev{M.S.}
\department{Computer Science}
\gradmonth{August}
\gradyear{2018}
\author{Berkeley Clint Hall}
\authorcaps{BERKELEY CLINT HALL}
\thesischair{Dr. Mitch Parry}
\deptchair{Dr. James T. Wilkes}
\dean{Max C. Poole, Ph.D.}

\begin{document}
    \begin{preliminary}
        \maketitle
    	\begin{abstract}
            The automatic prediction of prosodic stress (syllables which have special prominence in the context of a sentence or phrase) and other prosodic features of text has received some scholarly attention due to its usefulness for text to speech engines.
            But prosodic stress recognition has other applications as well.
            Proper chanting of a text requires the recognition of prosodic stresses as well.
            Consequently, there are publications of the psalms available which have the prosodic stresses marked in print.
            Psalm markup for the purpose of chanting is called "pointing." The goal of this project is to take pointed psalms as data and take machine learning and natural language processing as tools to create an engine for prosodic stress recognition.
    	\end{abstract}
    	\tableofcontents
    \end{preliminary}
    \newlinestretch{2}

\section{Road map paragraph}

First, I will explain the term 'prosodic stress', discuss some of its near
synonyms in the literature, and explain why I chose to use it rather than those
other terms. Then I will discuss the background and motivation of this project
in psalm chanting in general and the structure of the Revised Grail Psalms in
particular. In the following section, I will discuss some of the literature on
automatic prediction of prosodic stress. Finally I will explain what I have
already done on this project, including the tools and methods used, and what I
plan to do for the rest of the project.

\section{Prosodic Stress} 

When you look up a polysyllabic word in a dictionary,
you will see in the pronunciation guide that one or two of its syllables are to
receive more emphasis or stress than others. The relatively greater degree of
emphasis of some syllables than others in the context of a word is called
"lexical stress"~\cite{crystal_dictionary_2008}. Often a dictionary will list a
primary and a secondary stress for a polysyllabic word.

Part of natural, non-monotonous speech involves applying greater emphasis to some words than others in the contexts of larger units of text, such as sentences or phrases.
It is typically the syllable of primary stress in the word that receives most of this extra emphasis~\cite{jurafsky_speech_2009}.
The increased emphasis of a word or syllable in the context of a sentence, phrase, or other multi-word unit has gone by many different names in the literature.
I will call it prosodic stress.<fn> Wikipedia Stress: Prosodic stress, or sentence stress, refers to stress patterns that apply at a higher level than the individual word – namely within a prosodic unit.
It may involve a certain natural stress pattern characteristic of a given language, but may also involve the placing of emphasis on particular words because of their relative importance (contrastive stress).</fn>  Given the way the word "prosodic" is used in the literature, it is clear that what I am interested in is a prosodic feature of a text.
As Logan De Ley puts is,

>> Prosody...
comprises all of the variables of timing, phrasing, emphasis, and intonation that speakers use to help convey aspects of meaning and to make their speech lively.
One of the challenges of oral reading is adding back the prosodic cues that are largely absent from written language.~\cite{de_ley_why_2012}

However, Taylor would say that more it would be more proper to call what I am interested in "suprasegmental prominence" rather than "prosodic prominence", which he believes to be inscrutable from text alone.

And "stress" seems to be a fairly neutral word for accent, emphasis or prominence (but see Taylor 2009 who reserves the word 'stress' for lexical stress~\cite[p.117]{taylor_text--speech_2009}).

Sometimes the different used for prosodic stress mark out theoretical differences among their users.
For example there is a controversy concerning the acoustic features that are most responsible for the impression of emphasis.
Those who believe pitch (as opposed to volume or duration) is responsible seem more likely to talk about stress as 'accent'\~cite{crystal_dictionary_2008} or even 'pitch accent'~\cite{brenier_automatic_2008, hirschberg_pitch_1993}.
Sometimes authors are interested in a narrower phenomenon than prosodic stress, like "contrastiv accent"~\cite{theune_contrastive_1997} or a broader phenomenon like "prosody contour"~\cite{fernandez_prosody_2014}.
But sometimes the various terms do not seem to signal any sort of meaningful distinction, but are just a needless multiplication of terms.
We have the terms "prosodic prominence"~\cite{jurafsky_speech_2009}, "word prominance"~\cite{widera_prediction_1997}, "intonational prominance"~\cite{nakatani_computational_1997}.


\section{Background on Chanting in general}

In order to sing a text to an ordinary song tune, the text must conform to the proper meter for the tune.
The meter of a poetic text is the number of lines per stanza, syllables per line, and the number and position of prosodic stresses in each line.
For example, there are many common meter tunes and texts.
In common meter, stanzas have four lines.
Lines 1 and 3 have 8 syllables each and lines 2 and 4 have 6 syllables each.
In each line, the syllables follow the *unstressed, stressed, unstressed, stressed* pattern known as iambic.
Any common meter text can be sung to any common meter tune (try switching the words and tunes for "Amazing Grace" and "House of the Rising Sun").

"The psalms are meant to be sung"~\cite[p.xi]{the_benedictine_monks_of_conception_abbey_revised_2010}.
Their Hebrew name, "tehillim", means "songs of praise" and the Greek "psalmoi" means "songs to be sung to the sound of the harp."[General Instruction on the Liturgy of the Hours, no 103.
Cited by rgp Forward xi]  Ordinary song tunes can be used for singing the psalms, but only if a special metrical translation of the psalms is used.
There are many books of metrical psalms available; however, neither the original Hebrew of the psalms nor the translations found in any major edition of the bible are metrical in the relevant sense.

For well over a thousand years there has been a tradition of chanting the psalms.
Psalm chanting does not depend on any special translation.
Any text can be chanted.
Rather than using a musical tune in the sense we normally think of, a chanter sings a text to a rhythmless **tone**.
Instead pairing notes with syllables based on their position in a meter, a chant tone pairs notes with syllables in a text based on where the prosodic stresses fall in a natural reading of the text.

The simplest chant tones only require a change in pitch on the last stressed syllable in each line.
A very simple tone instructs us to consider the lines in pairs, and to go up in pitch by a whole step on the last stressed syllable of the first line, then back down to the starting pitch on the last syllable of the second line.
A tone might also instruct a change of pitch on the note or two preceding the last stressed syllable.
More complex tones, such as those written by David Clayton, give instructions around the last two stressed syllables in each line.
The St.
Meinrad Abbey tones do not require lines be looked at in pairs, but rather in stanzas of 2 to 6 lines.
Only the last stressed syllable of each line matters for pitch change placement.


\section{Background on the Revised Grail Translation}

There is one other sort of psalm tone that will be particularly relevant to this study.
The Gelineau tones expect a certain number of prosodic stresses per line, though the number of syllables per line and lines per stanza follow no set pattern.
Apparently, the original Hebrew poetry of the psalms conformed to this sort of "accentual verse" or "sprung meter".
Joseph Gelineau was interested in reproducing these Hebrew rhythmic patterns for singing in modern vernacular languages~\cite[p.xvi]{the_benedictine_monks_of_conception_abbey_revised_2010}.
With collaborators, he reworked the psalter of *La Bible de Jérusalem* into accentual verse and published psalm tones to to which it could be sung [https://www.giamusic.com/bios/gelineau_joseph.cfm].
The Ladies of the Grail (a catholic organization of lay people) soon assembled a team to produce a similar translation in English [rgp xvi].
The product of their work was published as *The Grail Psalms* in 1963.
In an attempt to improve authenticity of translation while preserving the sprung meter [rgp xviii], the brothers of Conception Abbey undertook a revision of the Grail Psalms with was published as *Revised Grail Psalms* in 2010.

Two things make the *Revised Grail Psalms* (RGP) particularly important to this study.
Most sorts of psalm tones only concern themselves with the last stress in each line, so most publications of pointed psalms only include markup around the last prosodic stress in each line.
But since every prosodic stress in each line matters to Gelineau psalmody, there is an edition of the Revised Grail Psalms in which every stress in each line is marked.

Second, the publishers of the GIA Publications, Inc., the publishers of the *Revised Grail Psalms* have given me an electronic copy of them with the prosodic stresses marked.
Since pointed psalms of any sort are hard to come by in electronic form, this is a big deal.

The fact that the Revised Grail Psalms are fairly unique in their accentual verse structure, means that one could train classifiers on them which would be essentially useless (given that the RGP have all been pointed by hand).
However, if nothing specific to accentual verse is given to the classifier as a feature, then the classifier might perform well on other versions of the psalms as well.
On the other hand, it would be interesting to see if a classifier trained on the grail psalms with an eye towards its accentual verse features might be able to point other versions of the psalms for use with Gelineau tones.


\section{Other sources of data}

To explore these questions, we need data from other psalm translations.
Fortunately, I pointed a number of psalms from the Book of Common Prayer psalter (BCP) and the Psalms of the King James Version (KJV) for a web application I made a few years ago.
Although they are considerably smaller than the RGP data set (The RGP contains 5738 lines, while I have only pointed 370 and 380 lines of the BCP and KJV respectively), it should be sufficiently large to make some interesting comparisons.


\section{Background on prosodic stress recognition from text}

The ability to correctly predict the prosodic features of text is import for natural sounding text to speech (TTS) engines.
Consequently, a good deal of research has been devoted to the problem.

This realm of research is only fully opening up to me. Before data driven
approaches that use various forms of machine learning, a variety of rule based
systems were used to predict prosodic prominance. In 1993, Julia Hirschberg
wrote, "Most current test-to-speech systems generally use simple word-class
information for input text to determine which items to accent and which to
deaccent" (p1), where content words like verbs and nouns would be given an
accent and function words like prepositions and articles would not. In that
seminal paper Hirschberg goes on to show that more detailed part of speech
information could be used to reliably predicts about 75% of prosodic stresses
[p19]. To achieve higher predictive success, Hirschbirg developed a complex set
of manually derived rules involving "a hierarchical representation of the
attentional structure of the discourse... which supported inference of local and
global focus, as well as contrast" (p19). Deterministic application of her rules
were able to achieve a prediction rate of 98.3% on citation-form
sentences<fn>I'm not really sure what citation form sentences are. I know that
there is a corpora of them that Hirschberg used. I know these are two sentences
in the corpora: "Plow a young man" and "The large cow accidentally brushed
against the fence" [Hirschberg 1993], and that there were annotated based on
readings in which "intonation is monotonous and regular, and each sentence has
clearly been uttered with little influence from the semantic content of prior
utterances" [Hirschberg 2001]</fn>. More difficult sentences could be predicted
at a rate of 80-85% by deterministic application of her rules and by
classification and regression trees (CART) automatic classification techniques.

In later work, it was found that contrast, information content, and focus can be
effectively modeled by features that are much easier to extract such as N-grams
and tf-idf (Term-Frequency/Inverse Document Frequency) [Jurafsky and Martin
2009, citing Pan and Hirschberg, 2000 and Pan and McKeown, 1999].

Prosodic stress prediction is very complicated in compound noun phrases
(combinations of nouns and adjectives) [Taylor 2009, p137-8]. The seminal work
on that topic is Sproat 1994. Sproat considers both rule based and data-driven
statistical methods for assigning stress within noun phrases for TTS.

Every resource I find leads to several more.
It may be that every technique in machine learning has been applied to prediction of prosodic features of text.

Let me say a couple things that are relevant to the challenges I will face.  (1) it is much harder to predict prosodic prominence in spontaneous speech is than in written speech.
That counts in my favor. (2) the hardest things to predict are (a) prosodic prominence in noun phrases and (b) function word prominance (sometimes words like 'but' and 'and' get stressed, sometimes they don't, and it is difficult to predict when) [Taylor 2009, p137-9].


\section{What I have already done}

I have already taken my three data sets (RGP, BCP, and KJV) and transformed each into a common text based form with prosodic stresses marked orthographically with acute accent marks over the first vowel in the syllable of primary stress in each prosodically stressed word.
I have python code that relies on pandas, numpy, and nltk (Natural Language Tool Kit) to transform each data set into a data frame in which tokens are indexed by psalm, by line, and by sentence.

Further, I have extracted two features.
First, I have extracted the part of speech using nltk, which relies on the Penn Treebank part of speech (POS) parser as implemented in the nltk.
The fact that the nltk POS parser requires tokenized data divided into sentences explains why my data must be divided into tokens indexed by sentence.

Second, I have extracted a measure of token rarity.
I did this by calculating the number of occurrences of a given token in a data set.
Then comparing obtaining a comparative measure of the rarity of each token compared to its neighbors in the same line (this is why my tokens had to be indexed by line).
I looked at the number of times each token in a line occurs in the data set, finding the average and standard deviation.
Then a particular word in the line was given a score by subtracting the average occurrence count in the line from its occurrence count and dividing by the standard deviation for occurrence counts in the line.
My intuition here is that words that are rarer than their neighbors are more informative and more likely to be stressed.

I have used these two features to train a logistic regression classifier using python's sklearn library under the direction of Raschka 2015.
I have achieved surprisingly good predictive accuracy using only these two features.

 | ROC AUC
--- | ---
RGP | .864
BCP | .856
KJV | .759

I need to investigate why the outcome is so much worse with the KJV.
It might show a flaw in the parsing of the dataset.

\section{What I plan to do}

As I come to better understand classifiers that have been used for prosodic stress detection by others, I plan to apply them to my data sets to see and compare their performance.
I will test classifiers trained on other data sets on my data sets, and I will also train classifiers designed by others on a portion of each of my data sets and then test them on the remainders.
I will generate confusion matrices and ROC AUC scores for the results.

Also, I will develop my own classifiers as features occur to me and test them as well.
First, I would like to try adding the expectation of an equal number of prosodic stresses per line as a feature.
I expect it to dramatically improve prediction on RGP, and marginally improve prediction on KJV and BCP.

In keeping with the agile practices of frequent releases and negotiated scope contracts, I intend to quickly achieve a position where I have a functioning project with a comparison of classifiers, then add employ a greater variety of machine learning models as time permits.
The further I can extend my expertise, the better.

I plan to take my thesis project course in Spring or Fall of 2018.
When work on my project officially begins, I plan to start by attempting to implement Pan and Hirschberg's 2000 classifier using sklearn and testing its performance.
Then I will add my own features and test the performance again.
I would then learn to produce the figures needed for a report to compare those two classifiers, and I would produce a simple report based on the experiment.
I expect to be able to do this in about two weeks of work.
From there, I plan to attempt a MEMM, following Gregory and Altun 2004.
Eventually, I would love to attempt a Deep Recurrent Neural Network of the sort described in Fernandez et al 2014 or Fernandez et al 2015 (right now I have very little grasp on what those articles are talking about).

In the interim, I plan to improve my basic understanding of the computational linguistics and TTS by continuing to study Jurafsky and Martin 2009 and Taylor 2009.
My python skills and general understanding of how to implement machine learning models will continue to improve as a finish Raschka 2014 and the NLTK Book.
\addcontentsline{toc}{chapter}{Bibliography}
    \bibliographystyle{plain}
\bibliography{proposal}

\end{document}
